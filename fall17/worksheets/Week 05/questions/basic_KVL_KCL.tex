%Author: Anwar Baroudi
%Email: mabaroudi@berkeley.edu, shreyas.partha@berkeley.edu,ndilip@berkeley.edu

\qns{Introduction to KVL, KCL}
\begin{mdframed}
    KVL: Kirchhoff's Voltage Law says that the sum of voltage drops going around a closed network (starting and ending at the same point) is 0



    KCL: Kirchhoff's Current Law says that the sum of all currents flowing into a node are equal to the sum of all currents flowing out of a node. This is equivalent to saying that the sum of all currents flowing out of a node (representing flows in opposite direction as negative) is equal to the sum of all currents flowing into a node (again, representing flows in opposite direction as negative) which is equal to zero
\end{mdframed}

Consider the following circuit, with $R_1 = 2 \Omega$, $R_2 = 3 \Omega$, and $R_3 = 4 \Omega$:

\begin{center}
    \begin{circuitikz}
    \draw(0,4)
	to[V_=$12 V$,invert] ++(0,-3);
	
 	\draw(0,4)
 	to[short] node[ground]{} ++(4,0);
 	\draw(3,4)
	to[short] ++(0,-3)
	to[short] ++(-1,0)
	to[R,l=$R_2$] ++(-1,0)
	to[short] node[]{} ++(-1,0)
	to[short] ++(0,-1)
	to[R,l=$R_1$] ++(0,-1)
	to[short] node[]{} ++(0,-1)
	to[short] node[]{} ++(3,0)
	to[short] ++(0,1)
	to[R,l=$R_3$] ++(0,1)
	to[short] node[]{} ++(0,1);
    \end{circuitikz}
\end{center}

We will solve this picture using nothing but KCL and KVL in the following steps.
\begin{enumerate}

\qitem{
    First, draw a reference point (ground) on the circuit, and label all the nodes $V_1$, $V_2$, etc.
}

\ans{
    You should draw a ground on either side of the voltage source, and the 3 nodes (labelled any way) as shown below. We'll continue using this labeling convention for the rest of the problem.
   \begin{center} 
    \begin{circuitikz}
    \draw(0,4)
	to[V_=$12 V$,-*,invert] node[label={[font=\footnotesize]left:$V_1$}] {} ++(0,-3);
	
 	\draw(0,4)
 	to[short] node[ground]{} ++(4,0);
 	\draw(3,4)
	to[short,-*] node[label={[font=\footnotesize]right:$V_3$}] {} ++(0,-3)
	to[short] ++(-1,0)
	to[R,l=$R_2$] ++(-1,0)
	to[short] node[]{} ++(-1,0)
	to[short] ++(0,-1)
	to[R,l=$R_1$] ++(0,-1)
	to[short,-*] node[label={[font=\footnotesize]below:$V_2$}] {} ++(0,-1)
	to[short] node[]{} ++(3,0)
	to[short] ++(0,1)
	to[R,l=$R_3$] ++(0,1)
	to[short] node[]{} ++(0,1);
    \end{circuitikz}
    \end{center}
}

\qitem{
    Next, draw the +s and -s on the resistors.
}

\ans{
    While there is no incorrect way to do this (as long as you are consistent with the equations in future parts), the way of doing so that is consistent with the method used in this problem is: \\
    For $R_1$, have the + be on the northside and the - on the southside.\\
    For $R_2$, have the + be on the leftside and the - on the rightside. \\
    For $R_3$, have the + be on the southside and the - on the northside.
    
   \begin{center} 
    \begin{circuitikz}
    \draw(0,4)
	to[V,v=$12V$,-*,invert] node[label={[font=\footnotesize]left:$V_1$}] {} ++(0,-3);
	
 	\draw(0,4)
 	to[short] node[ground]{} ++(4,0);
 	\draw(3,4)
	to[short,-*] node[label={[font=\footnotesize]right:$V_3$}] {} ++(0,-3)
	to[short] ++(-1,0)
	to[R,l=$R_2$,v<=$ $] ++(-1,0)
	to[short] node[]{} ++(-1,0)
	to[short] ++(0,-1)
	to[R,l=$R_1$,v>=$ $] ++(0,-1)
	to[short,-*] node[label={[font=\footnotesize]below:$V_2$}] {} ++(0,-1)
	to[short] node[]{} ++(3,0)
	to[short] ++(0,1)
	to[R,l=$R_3$,v>=$ $] ++(0,1)
	to[short] node[]{} ++(0,1);
    \end{circuitikz}
    \end{center}
}
\meta{
    Make sure to go over what difference is made by the + and - location, namely the idea that Ohm's law is about $\Delta$V, not just V.
}

\qitem{
    Finally, draw arrows indicating the direction of current, labeling them as $I_1$, $I_2$, etc.
}

\ans{
    Once again, as long as you are consistent for the next part, there isn't a wrong answer, but our equations use: \\
    $I_1$ is an arrow going south between the source and $V_1$. \\
    $I_2$ is an arrow going east between $V_1$ and $V_3$ around $R_2$.\\
    $I_3$ is an arrow going east between $V_2$ and $V_3$ around $R_3$.
    
    Our drawn on circuit should now look something like:
    
   \begin{center} 
    \begin{circuitikz}
    \draw(0,4)
	to[V,v=$12V$,-*,invert] node[label={[font=\footnotesize]left:$V_1$}] {} ++(0,-3);
	
 	\draw(0,4)
 	to[short] node[ground]{} ++(4,0);
 	\draw(3,4)
	to[short,-*] node[label={[font=\footnotesize]right:$V_3$}] {} ++(0,-3)
	to[short, i<=$i_2$] ++(-1,0)
	to[R,l=$R_2$,v<=$ $] ++(-1,0)
	to[short] node[]{} ++(-1,0)
	to[short] ++(0,-1)
	to[R,l=$R_1$,v>=$ $] ++(0,-1)
	to[short,-*,i=$i_1$] node[label={[font=\footnotesize]below:$V_2$}] {} ++(0,-1)
	to[short, i=$i_3$] node[]{} ++(3,0)
	to[short] ++(0,1)
	to[R,l=$R_3$,v>=$ $] ++(0,1)
	to[short] node[]{} ++(0,1);
    \end{circuitikz}
    \end{center}
}

\qitem{
Now write all the equations given by Ohm's law, KCL, and KVL that we will be using.
}

\ans{
    Let's write the equations categorized by the rule that gives it:
    First, those we can gain by KVL:\\
    \begin{align*}
        V_1 &= 12v \\
        V_3 &= 0v 
    \end{align*}
    Next, those we gain by Ohm's Law (i.e. $Delta V = RI$):
    \begin{align}
        V_1 - V_2 &= 2I_3 \\
        V_2 - V_3 &= 4I_3 \\
        V_1 - V_3 &= 3I_2 
    \end{align}
    Finally, we get an equation from KCL:
    \begin{align*}
        I_1 &= I_2 + I_3
    \end{align*}
}

\qitem{
    Now that you have all equations written out, solve them to find all unknown voltages and currents.
}
\ans{
    Let's begin by substituting $V_1$ and $V_3$, whose values we know, into equations (1)-(3):
    \begin{align}
        12 - V_2 &= 2I_3 \\
        V_2 - 0 &= 4I_3 \\
        12 - 0 &= 3I_2 
    \end{align}
    We see two equations set values equal to $I_3$. Let's scale them and set them equal to each other:
    \begin{align*}
        \frac{12 - V_2}{2} &= I_3 &\text{divide equation (4) by 2}\\
        \frac{V_2}{4} &= I_3 &\text{divide equation (5) by 4}\\
        \frac{12 - V_2}{2} &= \frac{V_2}{4} &\text{set the above two equations equal to each other}\\
        24 - 2V_2 &= V_2 &\text{multiply both sides by 4}\\
        V_2 &= 8v \\
    \end{align*}
    We now have all voltages. For current, let's plug these values back into equation (6) to get $I_3$ and $I_2$.
    \begin{align*}
        V_2 &= 4I_3 \\
        I_3 &= 2 \\
        3I_2 &= 12 \\
        I_2 &= 4
    \end{align*}
    And finally, we plug in these two into our KCL equation: 
    \begin{align*}
        I_1 &= I_2 + I_3 \\
        I_1 &= 6
    \end{align*}
    And with that we have now solved for all unknown values, getting:
    \begin{align*}
        V_1 &= 12\\
        V_2 &= 8 \\
        V_3 &= 0 \\
        I_1 &= 6 \\
        I_2 &= 4 \\
        I_3 &= 2
    \end{align*}
}

\end{enumerate}
