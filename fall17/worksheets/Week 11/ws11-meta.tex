\documentclass[11pt]{article}
\usepackage{../../ee16}
\usepackage{../../markup}
\usepackage{tikz}
\usetikzlibrary{calc,shapes.geometric,arrows,automata}
\usepackage{color,hyperref,listings,enumitem}
\usepackage{algorithm}
\usepackage{algpseudocode}
\usepackage{tkz-euclide}
\usepackage{physics}
\usepackage{pgfplots}
\usepackage[american,siunitx]{circuitikz}
\lstset{basicstyle=\ttfamily}
\newcommand{\fillin}[1]{\underline{\hskip #1}}
\newcommand{\doublehrule}{\hrule \vskip 0.02in \hrule}
\newcommand*\circled[1]{\tikz[baseline=(char.base)]{`'
  \node[shape=circle,draw,inner sep=2pt] (char) {#1};}}

\newcommand{\sol}[1]{{\color{blue} \textbf{Meta: } #1}} % solutions in blue
\newcommand{\meta}[1]{{\color{blue} \textbf{Meta: } #1}}
%\newcommand{\sol}[1]{} % no solution sketches
jos
\definecolor{blueish}{rgb}{0.7,0.1,.7}
%\newcommand{\ans}[1]{} % no numeric solutions
\newcommand{\ans}[1]{{\color{blueish} \textbf{Answer: } #1}} % numeric solutions

%\newcommand{\solans}[1]{}
\newcommand{\solans}[1]{{\color{blueish} \textbf{Answer: } #1}} %appears in both; use with caution

\begin{document}

\def\title{Worksheet 1}

\newcommand{\qitem}{\qpart\item}

\renewcommand{\labelenumi}{(\alph{enumi})} % change default enum format to (a)
\renewcommand{\theenumi}{(\alph{enumi})} % fix reference format accordingly.
\renewcommand{\labelenumii}{\roman{enumii}.} % second level labels.
\renewcommand{\theenumii}{\roman{enumii}.}

\maketitle

\vspace{0.5em}


\begin{qunlist}

% Author: Mudit Gupta
% Email: mudit+csm16a@berkeley.edu

\qns{Series equivalence... or not?}

\sol{Prereq: The fact that Q=CV, and the series and parallel equivalence formula for capacitors. \\
Description: Problem shows that capacitors sometimes look like they're in series but they're not.}

\begin{enumerate}
\qitem
Consider the following 2 circuits. What is the charge on the positive and negative plates of the two capacitors?

\begin{center}
\begin{circuitikz}
\draw(0,0) 
	to[short] ++(3,0)
	to[C=$C_1$, v<=$ $] ++(0,3)
	to[short] ++(-3,0)
	to[V=$V_s$] ++(0,-3)
	to[short] node[ground] {} ++(0,-1);

\draw(6,0)
	to[short] ++(3,0)
	to[C=$C_2$, v<=$ $] ++(0,3)
	to[short] ++(-3,0)
	to[V=$V_s$] ++(0,-3)
	to[short] node[ground] {} ++(0,-1);

\end{circuitikz}
\end{center}


\ans{
	The charge on the positive plate of $C_1$ is $C_1V_s$. The charge on the negative plate is then $-C_1V_s$. \\
	The charge on the positive plate of $C_2$ is $C_2V_s$. The charge on the negative plate is then $-C_2V_s$.
}

\qitem{
	Now consider that we first cut the capacitors off from their voltage sources and the ground nodes as such:

	\begin{center} 
	\begin{circuitikz}[line width=1pt]
	\draw(0,0)
		to[short] ++(0,0.75)
		to node[left, color=blue]{$-C_1V_s$} ++(0,0.25)
		to[C, l_=$C_1$, v<=$ $] ++(0,1)
		to node[left, color=red]{$C_1V_s$} ++(0,0.25)
		to[short] ++(0,0.75);
	\draw(3,0)
		to[short] ++(0,0.75)
		to node[left, color=blue]{$-C_2V_s$} ++(0,0.25)
		to[C, l_=$C_2$, v<=$ $] ++(0,1)
		to node[left, color=red]{$C_2V_s$} ++(0,0.25)
		to[short] ++(0,0.75);
	\end{circuitikz}
	\end{center}

	Next, we will connect these two capacitors as such:

	\begin{center}
	\begin{circuitikz}
	\draw(0,0)
		to[short] ++(1,0)
		to node[above, color=red] {$C_1V_s$} ++(0.25, 0)
		to[C, l_=$C_1$, v=$ $] ++(1,0)
		to node [above, color=blue] {$-C_1V_s$} ++(0.25,0)
		to[short] ++(1,0)
		to node[above, color=red] {$C_2V_s$} ++(0.25,0)
		to[C, l_=$C_2$, v=$ $] ++(1,0)
		to node[above, color=blue] {$-C_2V_s$} ++(0.25,0)
		to[short] ++(1,0);
	\end{circuitikz}
	\end{center}

	Question: Can the charges on the positive plate of the capacitor $C_1$ move?
}

\ans{
	No, because charges cannot jump across the plates of a capacitor and there is no path for these charges to escape.
}

\qitem{What about the charges on the negative plate of $C_1$ and the positive plate of $C_2$?}

\ans{
	In theory, these charges could redistribute... but look at the answer to the parts below.
}

\qitem{What about the charges on the negative plates of $C_2$?}

\ans{These also cannot move similar to the charges on the positive plate of $C_1$.}

\qitem{Here is a fundamental fact: If a capacitor's positive plate has $x$ charge, then the negative plate must have $-x$ charge! 
\\ Question: These two capacitors look like they're in series, but we know they don't have the same charge. Show how in this case, the series capacitance formula doesn't apply.}

\ans{The capacitors 'look' like they're in series, and they are if looked at as electrical components. However, let's go through the derivation of the 'series equivalence formula' for capacitors.\\

\textbf{Derivation}:
$$V = V_1 + V_2 + \ldots + V_n$$ where $V$ is the voltage drop across the branch of capacitors in series, and $V_i$ are the individual voltage drops.
$$\dfrac{Q_{eq}}{C_{eq}} = \dfrac{Q_1}{C_1} + \dfrac{Q_2}{C_2} + \ldots + \dfrac{Q_n}{C_n}$$ where $Q_{eq}$ is the charge on the equivalent capacitor, $C_{eq}$ is the equivalent capacitance, $Q_i$ is the charge on each individual capacitor and $C_i$ is the capacitance of each individual capacitor. \\
\textbf{Since we know that the charge on each individual capacitor is the same and the charge on the equivalent capacitor is also equal to this charge, $Q_{eq} = Q_1 = Q_2 = \ldots = Q_n = Q$ (say)} 
$$\dfrac{1}{C_{eq}} = \dfrac{1}{C_1} + \dfrac{1}{C_2} + \ldots + \dfrac{1}{C_n}$$ 

Notice that in this derivation we \textit{assume} that the charge on capacitors in series is the same, which leads to the formula. So capacitors having the same charge $\implies$ capacitors in series equivalence formula holds, not capacitors as components in series $\implies$ the series equivalence formula holds. The formula holds if and only if the capacitors are either discharged to begin with, or have the same charge on them to begin with!! \\

Intuitively, in this case, the capacitors do not have the same charge because the positive charges from $C_1$ and the negative charges from $C_2$ cannot move. This \textbf{forces} the negative charges of $C_1$ to stay where they are the positive charges of $C_2$ to stay where they are!! 
}

\sol{Mentors: please make sure to go through the derivation. This part is so important that it must be drilled in. \\}


\end{enumerate}
% Author: Mudit Gupta
% Email: mudit+csm16a@berkeley.edu

\qns{Pitfall Problem}

\sol{Prereq: Best placed right after the failure\_cap\_series\_equivalence.tex \\
Description: Once students can do this, they can do every capacitor charge sharing problem (not necessarily those that include op amps too though)}

\begin{enumerate}
\qitem Consider the following circuit in $\phi_1$. Assume that all capacitors are initially discharged. Find out the charge on each capacitor in this phase.
\begin{center}
\begin{circuitikz}
  % Phase 1 circuit
  \draw (0,0)
  to[V=$V_s$, invert] (0,2) % The voltage source
  to[short] (2,2)
  to[C=$C_1$, v=$ $] (2,0) % Capacitor C_1
  to[short] (0,0);

  \draw (2,2)
  to[short] (4,2)
  to[C=$C_2$, v<=$ $] (4,0)
  to[short] (2,0);

  \draw (4,2)
  to[short] (6,2)
  to[C=$C_3$, v=$ $] (6,0)
  to[short] (4,0);

  \draw (0,0) to[short] node[ground] {} (0,-1); %Mark ground at (0,-1)
  
\end{circuitikz}
\end{center}

\ans{
	$$Q_{C_1, \phi_1} = C_1V_s$$
	$$Q_{C_2, \phi_1} = -C_2V_s$$
	$$Q_{C_3, \phi_1} = C_3V_s$$

	Note: What does it mean when we say that the charge on a capacitor is \textit{negative}? It means that \textbf{on the positive plate of the capacitor, there is negative charge.} And since a capacitor has equal and opposite charge on each plate, the negative plate then has positive charge. 
}

\sol{
	Make sure to stress the 'Note'. This is super important and is a misconception that students carry through the course. $$Q=CV=C(V_+ - V_-)$$ where $V_+$ and $V_-$ are plates you arbitrarily mark as $+$ and $-$. It is obviously possible that you mark them incorrectly, and that there is actually negative charge on the positive plate and vice versa. If you use $Q=CV$ to calculate the charge on the capacitor, this always calculates the charge on the $\textbf{positive}$ plate of the capacitor. If this quantity is negative, then there is negative charge on the plate you arbitrarily marked positive, and positive charge on the plate you arbitrarily marked positive. $\textbf{This does not mean that you need to fix something in your circuit.}$\\ $\textbf{This is fine. Just be consistent in the next phases}$.

}


\qitem 
Assume that $\phi_1$ has taken place, and that the capacitors are then moved to the following configuration in $\phi_2$. Calculate the charge across each capacitor in $\phi_2$.

\begin{center}
\begin{circuitikz}
% Phase 2 circuit

	\draw (0,0)
	to[C=$C_1$, v<=$ $] (0,2)
	to[short] (2,2)
	to[C=$C_2$, v<=$ $] (2,0)
	to[short] (0,0);

	\draw (4, 0)
	to[C, l_=$C_3$, v<=$ $] (4,2)
	to[short] (6,2)
	to[short] (6,0)
	to[short] (4,0);

	\draw (0,0) to[short] node[ground] {} (0,-1); %Mark ground at (0,-1)

\end{circuitikz}
\end{center}

\ans{
	$\textbf{Step I: Write the voltage drop across each capacitor.}$ \\
	$\textit{If it cannot be determined, create variables till it can be determined.}$

	Circuit redrawn with unknown voltages marked. 

	\begin{center}
	\begin{circuitikz}
	\draw (0,0)
	to[C=$C_1$, v<=$ $] (0,2)
	to[short, l^=$\textcolor{red}{V_x}$] (2,2)
	to[C=$C_2$, v<=$ $] (2,0)
	to[short] (0,0);

	\draw (4, 0)
	to[C, l_=$C_3$, v<=$ $] (4,2)
	to[short, l^=$\textcolor{red}{V_y}$] (6,2)
	to[short] (6,0)
	to[short] (4,0);

	\draw (0,0) to[short] node[ground] {} (0,-1); %Mark ground at (0,-1)
	\end{circuitikz}
	\end{center}

	$$V_{C_1, \phi_2} = V_x - 0 = V_x$$
	$$V_{C_2, \phi_2} = 0 - V_x = -V_x$$
	$$V_{C_3, \phi_2} = V_y - V_y = 0$$


	$\textbf{Step 2: Write the charge on each capacitor.}$\\
	$\textit{Just use $Q=CV$ on the first step}$

	$$Q_{C_1, \phi_2} = C_1V_x$$
	$$Q_{C_2, \phi_2} = -C_2V_x$$
	$$Q_{C_3, \phi_2} = 0$$

	$\textbf{Step 3: Write the charge sharing equations on floating nodes.}$ \\
	$\textit{Floating nodes are those where charges cannot escape or enter.}$ \\
	$\textit{Node marked $V_x$ is the only floating node}$ \\

	\begin{equation} \label{ph2:chg_sharing}
	Q_{C_1, \phi_2} \textcolor{red}{-} Q_{C_2, \phi_2} = Q_{C_1, \phi_1} \textcolor{red}{-} Q_{C_2, \phi_1}
	\end{equation}

	$$C_1V_x - (-C_2V_x) = C_1V_s - (-C_2V_s) \implies V_x=V_s$$

	$\textbf{Final Step 4: Plug variable voltage into charge equations}$ \\

	$$Q_{C_1, \phi_2} = C_1V_x = C_1V_s$$
	$$Q_{C_2, \phi_2} = -C_2V_x = -C_2V_s$$
	$$Q_{C_3, \phi_2} = 0$$




}

\sol{
	Note: Make sure to explain the charge sharing equations properly. 

	\textbf{Pitfall number 1}: Students think that charge sharing equations always have positive signs on them. Something along the lines of $Q_{1,1}+Q_{2,1} = Q_{1,2} + Q_{2,2}$. Equation \ref{ph2:chg_sharing} shows that this is not the case. Explain clearly why there is a negative sign. This is because the positive plate of $C_1$ is connected to the negative plate of $C_2$. Charge is being redistributed by $Q_{C_1}$ and $-Q_{C_2}$ in the two phases. 

	\textbf{Pitfall number 2}: Students think that charge sharing always happens between all capacitors. When you ask them for charge sharing equations, their gut reaction will be to say $Q_{C_1, \phi_1} + Q_{C_2, \phi_1} + Q_{C_3, \phi_1} = Q_{C_1, \phi_2} + Q_{C_2, \phi_2} + Q_{C_3, \phi_2}$. Show that them their gut reaction is wrong -- because clearly $C_3$ is not involved in charge sharing. Also show them that the plates affect the signs for charge sharing, it isn't always a plus sign as shown in Equation \ref{ph2:chg_sharing}.

	\textbf{Pitfall number 3}: Many students don't realize that the voltage drop across $C_3$ is 0. They think that it will be the same as before and it will hold the same charge as before. This is not the case, as the math shows. Intuitively, this is not the case because + charges flow to recombine with the - charges resulting in 0 charge on the capacitor. 
}


\qitem

Assume that $\phi_2$ has taken place, and that the capacitors are then moved to the following configuration in $\phi_3$. Calculate the charge across each capacitor in $\phi_3$.

\begin{center}
\begin{circuitikz}
% Phase 3 circuit

	\draw (0,-1) 
	to[V=$V_s$, invert] (0,0)
	to[C=$C_1$, v<=$ $] (0,4)
	to[short] (2,0)
	to[C=$C_2$, v=$ $] (2,4)
	to[short] (4,4)
	to[C=$C_3$, v=$ $] (4,0)
	to[short] node[ground] {} (4,-2);

	\draw (0,-1) to[short] node[ground] {} (0,-2); 

\end{circuitikz}
\end{center}

\ans{
	$\textbf{Step I: Write the voltage drop across each capacitor.}$ \\
	$\textit{If it cannot be determined, create variables till it can be determined.}$

	Circuit redrawn with unknown voltages marked. 

	\begin{center}
	\begin{circuitikz}
	\draw (0,-1) 
	to[V=$V_s$, invert] (0,0)
	to[C=$C_1$, v<=$ $] (0,4)
	to[short, l^=$V_a$] (2,0)
	to[C, l_=$C_2$, v=$ $] (2,4)
	to[short, l^=$V_b$] (4,4)
	to[C=$C_3$, v=$ $] (4,0)
	to[short] node[ground] {} (4,-2);

	\draw (0,-1) to[short] node[ground] {} (0,-2); 
	\end{circuitikz}
	\end{center}

	$$V_{C_1, \phi_3} = V_a - V_s$$
	$$V_{C_2, \phi_3} = V_a - V_b$$
	$$V_{C_3, \phi_3} = V_b$$


	$\textbf{Step 2: Write the charge on each capacitor.}$\\
	$\textit{Just use $Q=CV$ on the first step}$

	$$Q_{C_1, \phi_3} = C_1(V_a-V_s)$$
	$$Q_{C_2, \phi_3} = C_2(V_a-V_b)$$
	$$Q_{C_3, \phi_3} = C_3(V_b)$$

	$\textbf{Step 3: Write the charge sharing equations on floating nodes.}$ \\
	$\textit{Floating nodes are those where charges cannot escape or enter.}$ \\
	$\textit{Nodes marked $V_a$ and $V_b$ are the only floating nodes}$ \\

	\begin{equation} \label{ph3:chg_sharing1}
	Q_{C_1, \phi_3} \textcolor{red}{+} Q_{C_2, \phi_3} = Q_{C_1, \phi_2} \textcolor{red}{+} Q_{C_2, \phi_2}
	\end{equation}

	\begin{equation} \label{ph3:chg_sharing2}
	-Q_{C_2, \phi_3} \textcolor{red}{+} Q_{C_3, \phi_3} = -Q_{C_2, \phi_2} \textcolor{red}{+} Q_{C_3, \phi_2}
	\end{equation}

Plugging in values to Equation \ref{ph3:chg_sharing1}
	$$C_1(V_a-V_s) + C_2(V_a-V_b) = C_1V_s + (-C_2V_s)$$

Plugging in values to Equation \ref{ph3:chg_sharing2}
	$$-C_2(V_a-V_b) + C_3V_b = -(-C_2V_s) + 0$$

$\textbf{Final Step 4: Plug variable voltage into charge equations}$ \\
We now have two equations in two variables ($V_a, V_b$), and so we can solve for them. After that, we can plug those into Step 2, and find the charges on each capacitor. There is no need to do this step as the expressions aren't very neat. 
	

}
\sol {
	\textbf{Pitfall number 4}: Make sure to explain equations \ref{ph3:chg_sharing1} and \ref{ph3:chg_sharing2} very carefully. Students get confused specially on \ref{ph3:chg_sharing1} because in $\phi_2$, the positive plate of $C_1$ is connected to the negative plate of $C_2$. In $\phi_3$, the positive plates of both are connected to each other. The thing is that we only need to look at the current phase $\phi_3$'s configuration to write the charge sharing equations. Previous configs don't matter. One way to explain this is by talking about phases as a 2 step process -- disconnection and connection. First disconnect all capacitors, and all positive charges stay where they are and negative stay where they are. Now, if we connect the positive plate of two capacitors and if originally the positive and negative plates of two separate capacitors were connected earlier, the previous config doesn't matter. It is the two positive plate's charges that will need to stay conserved between phases. This is why the signs on \ref{ph3:chg_sharing1} will be the same on both the left hand side of the equation and the right hand side, and you get the signs by simply looking at $\phi_3$.
}





\end{enumerate}


% Author: Mudit Gupta
% Email: mudit+csm16a@berkeley.edu

\qns{You're grounded for no damn reason}

\sol{Prereq: supernode.tex or understanding of what to do when voltage sources are in the way of doing nodal analysis. \\
Brief description: The problem walks through one circuit first with nodal analysis and then with superposition using different grounds to show that where you mark your ground makes no difference.}

In this class, we always say "choose your ground wherever you want". In this question, we will explore how our choice of ground can change our answer drastically!

\begin{enumerate}
\qitem{
	Consider the following circuit. In this, we have explicitly marked a ground node for you. We have also marked a direction for the polarities on the resistor. Using nodal analysis, find the voltage drop across each resistor.

	\begin{center}
	\begin{circuitikz}

	\draw(0,4)
	to[short] ++(0,-1)
	to[V_=$8V$] ++(0,-2)
	to[short] node[ground] {} ++(0,-2)
	to[short] ++(0,1)
	to[short] ++(2,0)
	to[R, l=$7k\Omega$, v=$ $, i=$ $] ++(0,4)
	to[R, l=$6k\Omega$, v=$ $, i=$ $] ++(-2,0);

	\draw(2,0)
	to[short] ++(2,0)
	to[I, l=$3mA$] ++(0,4)
	to[short] ++(-2,0);
	\end{circuitikz}
	\end{center}
}







\ans{
	We know the potential of the node marked ground. We also know the potential of the node connected to the positive terminal of the $8V$ source, it is $8V$. There is only one unknown potential, which is the top-left node. Let's write the equation for that node, and call the potential there $V_x$.

	$$\dfrac{V_x - 8V}{6k\Omega} = 3mA + \dfrac{0-V_x}{7k\Omega}$$
	Multiply by $42k\Omega$ on both sides
	$$\implies 7V_x - 56 = 126V + -6V_x$$
	$$\implies 13V_x = 182V$$
	$$V_x = 14V$$.

	Voltage drop across $R_{7k} = V_+ - V_- = 0 - 14V = -14V$. All this means is that the way I marked my polarities was incorrect, and that current actually flows the other way. \\
	Voltage drop across $R_{6k} = V_+ - V_- = 14V - 8V = 6V.$
}

\qitem{
	This time, we will mark our ground somewhere else and we will solve the circuit by superposition. 

	\begin{center}
	\begin{circuitikz}

	\draw(0,4)
	to[short] ++(0,-1)
	to[V_=$8V$] ++(0,-2)
	to[short] ++(0,-1)
	to[short] ++(2,0)
	to[R, l=$7k\Omega$, v=$ $, i=$ $] ++(0,4)
	to[R, l=$6k\Omega$, v=$ $, i=$ $] ++(-2,0);

	\draw(2,0)
	to[short] ++(2,0)
	to[I, l=$3mA$] ++(0,4)
	to[short] ++(-2,0)
	to[short] node[ground, rotate=180] {} ++(0,1);
	\end{circuitikz}
	\end{center}

}

\ans{
	This time, we have two unknown potentials to solve for. One at the + end of the voltage source, and one at the negative. Let's get to it. Let's call them $V_a$ and $V_b$ respectively.\\ \\ 

	For superposition, draw the circuit with only the voltage source, by open circuiting the current source. Let the potentials from this case be $V_{a_1}$ and $V_{b_1}$ respectively. 


			\begin{center}
	\begin{circuitikz}

	\draw(0,4)
	to[short] ++(0,-1)
	to[V_=$8V$] ++(0,-2)
	to[short] ++(0,-1)
	to[short] ++(2,0)
	to[R, l=$7k\Omega$, v=$ $, i=$ $] ++(0,4)
	to[R, l=$6k\Omega$, v=$ $, i=$ $] ++(-2,0);

	\draw(2,0)
	to[short] ++(2,0)
	to[open] ++(0,4)
	to[short] ++(-2,0)
	to[short] node[ground, rotate=180] {} ++(0,1);
	\end{circuitikz}
	\end{center}

	Consider the $V_a$ node. We don't know the current through the voltage source, so we use our super node trick. Consider $V_{a_1}$ and $V_{b_1}$ together. One equation is that $$V_{a_1} - V_{b_1} = 8V$$
	The second equation is that \textbf{Sum I into $V_{a_1}$, $V_{b_1}$ = Sum I out of $V_{a_1}$, $V_{b_1}$}

	$$\dfrac{0-V_{a_1}}{6k\Omega} = \frac{V_{b_1}-0}{7k\Omega}$$

	We know that $V_{a_1} = V_{b_1} + 8V$, so multiplying by $42k\Omega$

	$$-7V_{a_1} = 6V_{b_1} \implies -7V_{b_1} - 6V_{b_1} - 56V = 0 => V_{b_1} = \frac{-56}{13}V$$
	$$V_{a_1} = 8V + \frac{-56}{13}V = \frac{48}{13}V$$


	Now, short circuit the voltage source. Let the potentials here be $V_{a_2}$ and $V_{b_2}$ respectively. 
	\begin{center}
	\begin{circuitikz}

	\draw(0,4)
	to[short] ++(0,-1)
	to[short, *-*] ++(0,-2)
	to[short] ++(0,-1)
	to[short] ++(2,0)
	to[R, l=$7k\Omega$, v=$ $, i=$ $] ++(0,4)
	to[R, l=$6k\Omega$, v=$ $, i=$ $] ++(-2,0);

	\draw(2,0)
	to[short] ++(2,0)
	to[I, l=$3mA$] ++(0,4)
	to[short] ++(-2,0)
	to[short] node[ground, rotate=180] {} ++(0,1);
	\end{circuitikz}
	\end{center}

	First, observe that $V_{a_2} = V_{b_2}$. 
	Let's solve for the unknown node. 

	$$3mA + \dfrac{V_{a_2} - 0}{7k\Omega} = \dfrac{0 - V_{a_2}}{6k\Omega}$$
	$$126V + 6V_{a_2} = -7V_{a_2}$$
	$$V_{a_2} = \dfrac{-126}{13} = V_{b_2}$$

	Finally, $$V_a = V_{a_1} + V_{a_2} = \dfrac{48}{13} + \dfrac{-126}{13} = \dfrac{-78}{13} = -6V$$
	$$V_b = V_{b_1} + V_{b_2} = \dfrac{-56}{13}V + \dfrac{-126}{13}V = \dfrac{-182}{13}V = -14V$$

	What is the voltage drop across $R_{6k}$ in the original circuit? $V_+ - V_- = 0 - V_a = 0 - (-6)V = 6V$.\\
	What is the votlage drop across $R_{7k}$ in the original circuit $V_+ - V_- = V_b - 0 = -14V - 0 = -14V$.\\
}

\sol{
	Feel free to go ahead and mark the third node that we haven't marked as ground yet and do the problem once again. You can do it quickly. 
}

\qitem{
	In an exam, you're solving a mechanical problem with resistors, voltage sources and current sources, and it asks you to find the voltage drop across one of the resistors. The TAs forgot to mark a ground on the circuit. Do you ask for a clarification and point out that they've made a mistake?
}

\ans{
	The TAs haven't made a mistake. Where you mark your ground \textbf{DOES NOT MATTER}. 
}

\sol{
 Make sure that students realize \textbf{the ground DOES NOT MATTER}. Why? Because you measure the DROP, i.e., DIFFERENCE IN POTENTIAL. Yes, the potentials at each node are different, but that doesn't matter, we never measure it. Show them that we used a \textbf{different technique and a different ground}, and our answer still didn't change. Say it a million times if you have to... 

Fun fact: I wrote this problem on the day after Summer 17 MT2. One student legit asked me this during the exam, which motivated me to write this whole problem.}
\end{enumerate}
% Author: Mudit Gupta
% Email: mudit+csm16a@berkeley.edu

\qns{Supernode}

\sol{Prereq: Best placed after a super easy nodal analysis problem. \\
Description: This shows how to deal with multiple voltage sources that don't share a ground when doing nodal analysis since we don't know how much current goes through a voltage sources.}

In this question, we will explore how to deal with multiple voltage sources when doing nodal analysis. Throughout the problem, assume that $R_1 = 1 \Omega, R_2 = 2 \Omega, R_3 = 4 \Omega, R_4 = 4 \Omega, V_1 = 1 \mathrm{V}, V_2 = 2 \mathrm{V}, V_3 = 4 \mathrm{V}$. While it is good practice to symbolically solve problems before substituting specific numbers, for superposition it is often easiest to substitute before superposing the answers.

\begin{center}
\begin{circuitikz}

\draw(0,0) %Node A
	to[short] ++(1,0)
	to[short] ++(-1,0)
	to[short] ++(0,1)
	to[short] ++(0,-1)
	to[short] node[ground] {} ++(0,-1);

\draw(0,2) %Node B
	to[short] ++(0,1)
	to[short] ++(1,0);

\draw(2,3) %Node C
	to[short] ++(2,0)
	to[short] ++(-1,0)
	to[short] ++(0,-1);

\draw(5,3)
	to[short] ++(1,0)
	to[short] ++(0,-1);

\draw(6,1)
	to[short] ++(0,-1)
	to[short] ++(-1,0);

\draw(2,0)
	to[short] ++(2,0)
	to[short] ++(-1,0)
	to[short] ++(0,1);

\draw(1,0) to[R=$R_1$] ++(1,0);
\draw(1,3) to[R=$R_2$] ++(1,0);
\draw(4,3) to[V=$V_2$] ++(1,0);
\draw(4,0) to[V=$V_3$] ++(1,0);
\draw(3,2) to[R=$R_3$] ++(0,-1);
\draw(6,2) to[R=$R_4$] ++(0,-1);
\draw(0,1) to[V=$V_1$,invert] ++(0,1);

\end{circuitikz}
\end{center}

\begin{enumerate}
\qitem{Mark all the nodes. If you know the potential at the node, write down the value next to the node. If you don't know the value, then assign a variable for the potential.}

\ans{

$V_a, V_b, V_c, V_d$ are variables. We know the potentials at the node marked ground and at the node marked $V_1$. 
	\begin{center}
\begin{circuitikz}[line width=1pt]

\draw[color=brown](0,0) %Node A
	to[short, l^=$0V$] ++(0,0)
	to[short] ++(1,0)
	to[short] ++(-1,0)
	to[short] ++(0,1)
	to[short] ++(0,-1)
	to[short] node[ground] {} ++(0,-1);

\draw[color=green](0,2) %Node B
	to[short] ++(0,1)
	to[short, l^=$V_1$] ++(0,0)
	to[short] ++(1,0);

\draw[color=red](2,3) %Node C
	to[short] ++(2,0)
	to[short] ++(-1,0)
	to node[above]{$V_a$} ++(0,0)
	to[short] ++(0,-1);

\draw[color=blue](5,3)
	to[short] ++(1,0)
	to[short, l_=$V_b$] ++(0,0)
	to[short] ++(0,-1);

\draw[color=magenta](6,1)
	to[short] ++(0,-1)
	to[short, l_=$V_c$] ++(0,0)
	to[short] ++(-1,0);

\draw[color=orange](2,0)
	to[short] ++(2,0)
	to[short] ++(-1,0)
	to node[below] {$V_d$} ++(0,0)
	to[short] ++(0,1);

\draw(1,0) to[R=$R_1$] ++(1,0);
\draw(1,3) to[R=$R_2$] ++(1,0);
\draw(4,3) to[V=$V_2$] ++(1,0);
\draw(4,0) to[V=$V_3$] ++(1,0);
\draw(3,2) to[R=$R_3$] ++(0,-1);
\draw(6,2) to[R=$R_4$] ++(0,-1);
\draw(0,1) to[V=$V_1$,invert] ++(0,1);


\end{circuitikz}
\end{center}

}

\qitem {Mark current directions arbitrarily and corresponding polarities on each resistor. Note that if current goes from left to right, then the left side of the resistor is to be marked + and the right side must be marked -. This is the passive sign convention.}

\ans{


\begin{center}
\begin{circuitikz}[line width=0.5pt]

	\draw[color=brown](0,0) %Node A
	to[short, l^=$0V$] ++(0,0)
	to[short] ++(1,0)
	to[short] ++(-1,0)
	to[short] ++(0,1)
	to[short] ++(0,-1)
	to[short] node[ground] {} ++(0,-1);

\draw[color=green](0,2) %Node B
	to[short] ++(0,1)
	to[short, l^=$V_1$] ++(0,0)
	to[short] ++(1,0);

\draw[color=red](2,3) %Node C
	to[short] ++(2,0)
	to[short] ++(-1,0)
	to node[above]{$V_a$} ++(0,0)
	to[short] ++(0,-1);

\draw[color=blue](5,3)
	to[short] ++(1,0)
	to[short, l_=$V_b$] ++(0,0)
	to[short] ++(0,-1);

\draw[color=magenta](6,1)
	to[short] ++(0,-1)
	to[short, l_=$V_c$] ++(0,0)
	to[short] ++(-1,0);

\draw[color=orange](2,0)
	to[short] ++(2,0)
	to[short] ++(-1,0)
	to node[below] {$V_d$} ++(0,0)
	to[short] ++(0,1);

\draw(1,0) to[R=$R_1$, v=$ $, l_=$\stackrel{i_1}{\longrightarrow}{R_1}$] ++(1,0);
\draw(1,3) to[R=$R_2$, v=$ $, l_=$\stackrel{i_2}{\longrightarrow}{R_2}$] ++(1,0);
\draw(4,3) to[V=$V_2$] ++(1,0);
\draw(4,0) to[V=$V_3$] ++(1,0);
\draw(3,2) to[R=$R_3$, v=$ $, l_=$\stackrel{i_3}{\downarrow}{R_3}$] ++(0,-1);
\draw(6,2) to[R=$R_4$, v=$ $, l_=$\stackrel{i_4}{\downarrow}{R_4}$] ++(0,-1);
\draw(0,1) to[V=$V_1$,invert] ++(0,1);

\end{circuitikz}
\end{center}
}

\sol{Might be a good idea to mark the nodes in the same fashion as the solutions for later parts!}

\qitem{
	Note that we define 4 nodes with unknown potentials. So we need 4 equations. Each of these nodes with unknown potential should give us one equation. \\
	Write the equation for the first node with unknown potential. \\
}
\ans{
	For the $V_a$ node, we have a problem. The problem is that we don't know what current enters the voltage source $V_2$. The fix here is to treat $V_a$ and $V_b$ combined as a super-node. Since we are considering 2 nodes at the same time, this 'supernode' better give us 2 equations! \\
	We can get equation from the fact that the voltage source exists. \begin{equation} \label{supernode:1}V_a - V_b = V_2 \end{equation} 
	For the other equation, let's consider:
	(Sum of current entering nodes $V_a$ and $V_b$) = (Sum of current leading nodes $V_a$ and $V_b$). 

	\textbf{Consider $V_a$}:
	\begin{itemize}
	\item $i_2$ enters $V_a$. 
	\item $i_3$ leaves $V_a$. 
\end{itemize}
	\textbf{Consider $V_b$}: 
	\begin{itemize}
	\item$i_4$ leaves $V_b$. 
\end{itemize}
	$$i_2 = i_3 + i_4$$
	\begin{equation} \label{supernode:2} \dfrac{V_1 - V_a}{R_2} = \dfrac{V_a - V_d}{R_3} + \dfrac{V_b - V_c}{R_4} \end{equation}
}
\sol{For equation ~\ref{supernode:2}, make sure students know that in the numerator, you have to write it as $V_+ - V_-$, depending on how you marked the polarities. But it is super important to mark the + where the current starts and - where it ends. This is passive sign convention. Make sure they do this right! Also, if students ask why (Sum of current entering nodes $V_a$ and $V_b$) = (Sum of current leading nodes $V_a$ and $V_b$) is something that we can use, show on the circuit that current cannot stay between $V_a$ and $V_b$. Any current that enters $V_a$ or $V_b$ must leave through $V_a$ or $V_b$}

\qitem{
	Write equations for all remaining unknown nodes. Solve for all the unknown potentials.
}

\ans{
	Nodes $V_c$ and $V_d$ remain. Let's start at $V_c$. Again, we don't know what current enters $V_3$, so we do the same thing. Treat the 2 ends of the voltage source, $V_c$ and $V_d$ as a super-node. One equation we get is \begin{equation} \label{supernode:3} V_d - V_c = V_3 \end{equation}
	The other equation comes from (Sum of current entering nodes $V_a$ and $V_b$) = (Sum of current leading nodes $V_a$ and $V_b$). \\

	\textbf{Consider $V_c$:} 
	\begin{itemize} \item $i_4$ enters. \end{itemize} 

	\textbf{Consider $V_d$:} 
	\begin{itemize} \item $i_3$ enters
	\item $i_1$ enters. \end{itemize}

	$$i_4 + i_3 + i_1 = 0$$
	\begin{equation} \label{supernode:4} \dfrac{V_b - V_c}{R_4} + \dfrac{V_a-V_d}{R_3} + \dfrac{0-V_d}{R_1} = 0 \end{equation}

	Equations (\ref{supernode:1}), (\ref{supernode:2}), (\ref{supernode:3}), (\ref{supernode:4}) are four equations for the four variables $V_a, V_b, V_c, V_d$ and can be solved uniquely. One fun fact is that these equations are guaranteed to be linearly independent as long as the circuit is not an impossible circuit (such as 2 current sources of differing values in series, or 2 voltage sources of different values in parallel).
	
	To summarize, we have:
	\begin{align*}
	    \begin{cases*}
	    V_a - V_b = V_2\\
	    \dfrac{V_1 - V_a}{R_2} = \dfrac{V_a - V_d}{R_3} + \dfrac{V_b - V_c}{R_4}\\
	    V_d - V_c = V_3\\
	    \dfrac{V_b - V_c}{R_4} + \dfrac{V_a-V_d}{R_3} + \dfrac{0-V_d}{R_1} = 0 \\
	    \end{cases*}
	    &\implies
	    \begin{cases*}
	    V_a - V_b = 2\\
	    \dfrac{1-V_a}{2} = \dfrac{V_a - V_d}{4} + \dfrac{V_b - V_c}{4}\\
	    V_d - V_c = 4\\
	    \dfrac{V_b - V_c}{4} + \dfrac{V_a - V_d}{4} + \dfrac{-V_d}{1}\\
	    \end{cases*}
	\end{align*}
	Solving the right set of equations, we find that $\boxed{V_a = \frac{1}{5}\text{V}, V_b = -\frac{9}{5}\text{V}, V_c=-\frac{18}{5}\text{V}, V_d=\frac{2}{5}\text{V}}$.
}

\qitem{
	Now we will explore solving the circuit using superposition. To recall the process, which 2 sources will be suppressed when looking at each of the 3 sources. Will the other 2 sources be treated as open circuits or short circuits?
}

\ans{
    For $V_1$, $V_2$ and $V_3$ will be shorted. For $V_2$, $V_1$ and $V_3$ will be shorted. For $V_3$, $V_1$ and $V_2$ will be shorted. When doing superposition and suppressing inputs, a voltage source is treated as a short and a current source is treated as an open.
    
    The reasoning for "zeroing" voltage/current sources is as follows: we define a voltage source as something that maintains a fixed voltage drop between the two terminals. If we "zero" that source, we are saying the voltage across the two terminals should be $0 \text{V}$. But that, by definition, has the same effect as connecting the two terminals with a wire, i.e. shorting it! So we say "zeroed" voltage sources are shorted. A similar argument leads us to find "zeroed" current sources are equivalent to open circuits.
}

\qitem{
	Draw the superposition circuit for $V_1$ and solve this circuit.
}

\ans{

    \begin{center}
    \begin{circuitikz}[line width=0.5pt]
    
    	\draw[color=brown](0,0) %Node A
    	to[short, l^=$0V$] ++(0,0)
    	to[short] ++(1,0)
    	to[short] ++(-1,0)
    	to[short] ++(0,1)
    	to[short] ++(0,-1)
    	to[short] node[ground] {} ++(0,-1);
    
    \draw[color=green](0,2) %Node B
    	to[short] ++(0,1)
    	to[short, l^=$V_1$] ++(0,0)
    	to[short] ++(1,0);
    
    \draw[color=red](2,3) %Node C
    	to[short] ++(2,0)
    	to[short] ++(-1,0)
    	to node[above]{$V_a$} ++(0,0)
    	to[short] ++(0,-1);
    
    \draw[color=blue](5,3)
    	to[short] ++(1,0)
    	to[short, l_=$V_b$] ++(0,0)
    	to[short] ++(0,-1);
    
    \draw[color=magenta](6,1)
    	to[short] ++(0,-1)
    	to[short, l_=$V_c$] ++(0,0)
    	to[short] ++(-1,0);
    
    \draw[color=orange](2,0)
    	to[short] ++(2,0)
    	to[short] ++(-1,0)
    	to node[below] {$V_d$} ++(0,0)
    	to[short] ++(0,1);
    
    \draw(1,0) to[R=$R_1$] ++(1,0);% v=$ $, l_=$\stackrel{i_1}{\longrightarrow}{R_1}$] ++(1,0);
    \draw(1,3) to[R=$R_2$] ++(1,0); %v=$ $, l_=$\stackrel{i_2}{\longrightarrow}{R_2}$] ++(1,0);
    \draw(4,3) to[short,l=$V_a \eq V_b$] ++(1,0);
    \draw(4,0) to[short,l=$V_c \eq V_d$] ++(1,0);
    \draw(3,2) to[R=$R_3$] ++(0,-1);%, v=$ $, l_=$\stackrel{i_3}{\downarrow}{R_3}$] ++(0,-1);
    \draw(6,2) to[R=$R_4$] ++(0,-1);%, v=$ $, l_=$\stackrel{i_4}{\downarrow}{R_4}$] ++(0,-1);
    \draw(0,1) to[V=$V_1$, invert] ++(0,1);
    
    \end{circuitikz}
    \end{center}
    
    \begin{align*}
    &\begin{cases*}
        \dfrac{V_a - V_1}{R_2} + \dfrac{V_a - V_d}{R_3} + \dfrac{V_b - V_c}{R_4} = 0 \\
    \dfrac{V_d}{R_1} + \dfrac{V_d - V_a}{R_3} + \dfrac{V_c - V_b}{R_4} = 0 \\
    V_a = V_b \\
    V_c = V_d 
    \end{cases*}
    &\implies
    \begin{cases*}
    \dfrac{V_a - V_1}{R_2} + \dfrac{V_a - V_d}{R_3} + \dfrac{V_a - V_d}{R_4} = 0 \\
    \dfrac{V_d}{R_1} + \dfrac{V_d - V_a}{R_3} + \dfrac{V_d - V_a}{R_4} = 0
    \end{cases*} 
    \end{align*}
    Solving for $V_a, V_b, V_c, \text{ and } V_d$ we find
    \begin{align*}
    \begin{cases*}
    V_a = V_b = \dfrac{(R_1 R_2 + R_1 R_4 + R_3 R_4)}{R} V_1\\
    V_d = V_c = \dfrac{R_1R_3 + R_1R_4}{R} V_1,
    \end{cases*}&&
    R \equiv R_1R_3+R_2R_3+R_1R_4+R_2R_4+R_3R_4
    \end{align*}
    Or, if you prefer solving with numbers first:
    \begin{align*}
        \begin{cases*}
        \dfrac{V_a - 1}{2} + \dfrac{V_a - V_d}{4} + \dfrac{V_a - V_d}{4} = 0 \\
        \dfrac{V_d}{1} + \dfrac{V_d - V_a}{4} + \dfrac{V_d - V_a}{4} = 0
        \end{cases*}
        &\implies
        \boxed{
        \begin{cases*}
        V_a = \frac{3}{5}\\
        V_b = \frac{3}{5}\\
        V_c = \frac{1}{5}\\
        V_d = \frac{1}{5}
        \end{cases*}}
    \end{align*}
}

\qitem{
	Draw the superposition circuit for $V_2$ and write the equation that characterizes this circuit.
}

\ans{
    \begin{center}
    \begin{circuitikz}[line width=0.5pt]
    
    	\draw[color=brown](0,0) %Node A
    	to[short, l^=$0V$] ++(0,0)
    	to[short] ++(1,0)
    	to[short] ++(-1,0)
    	to[short] ++(0,1)
    	to[short] ++(0,-1)
    	to[short] node[ground] {} ++(0,-1);
    
    \draw[color=green](0,2) %Node B
    	to[short] ++(0,1)
    	to[short, l^=$V_1$] ++(0,0)
    	to[short] ++(1,0);
    
    \draw[color=red](2,3) %Node C
    	to[short] ++(2,0)
    	to[short] ++(-1,0)
    	to node[above]{$V_a$} ++(0,0)
    	to[short] ++(0,-1);
    
    \draw[color=blue](5,3)
    	to[short] ++(1,0)
    	to[short, l_=$V_b$] ++(0,0)
    	to[short] ++(0,-1)
    	to[short] ++(0,-1);
    
    \draw[color=magenta](6,1)
    	to[short] ++(0,-1)
    	to[short, l_=$V_c$] ++(0,0)
    	to[short] ++(-1,0);
    
    \draw[color=orange](2,0)
    	to[short] ++(2,0)
    	to[short] ++(-1,0)
    	to node[below] {$V_d$} ++(0,0)
    	to[short] ++(0,1);
    
    \draw(1,0) to[R=$R_1$] ++(1,0);% v=$ $, l_=$\stackrel{i_1}{\longrightarrow}{R_1}$] ++(1,0);
    \draw(1,3) to[R=$R_2$] ++(1,0); %v=$ $, l_=$\stackrel{i_2}{\longrightarrow}{R_2}$] ++(1,0);
    \draw(4,3) to[V,l=$V_2$] ++(1,0);
    \draw(4,0) to[short,l=$V_c \eq V_d$] ++(1,0);
    \draw(3,2) to[R=$R_3$] ++(0,-1);%, v=$ $, l_=$\stackrel{i_3}{\downarrow}{R_3}$] ++(0,-1);
    \draw(6,2) to[R=$R_4$] ++(0,-1);%, v=$ $, l_=$\stackrel{i_4}{\downarrow}{R_4}$] ++(0,-1);
    \draw(0,1) to[short, l=$V_1 \eq 0$] ++(0,1);
    
    \end{circuitikz}
    \end{center}
    \begin{align*}
    \begin{cases*}
    \dfrac{V_1 - V_a}{R_2} + \dfrac{V_1 - V_d}{R_1} = 0 \implies V_a = -2 V_d\\
    \dfrac{V_d}{R_1} + \dfrac{V_d - V_a}{R_3} + \dfrac{V_c - V_b}{R_4} = 0 \implies 5V_d-V_a+V_c-V_b = 0\\
    V_c = V_d\\
    V_a = V_b + V_2 = V_b + 2\\
    \end{cases*} 
    \end{align*}
    \begin{align*}
    \begin{cases*}
    V_a = -2V_d\\
    6 V_d - 2V_a + 2 = 0 
    \end{cases*}
    &\implies
    \boxed{
    \begin{cases*}
    V_a = \dfrac{2}{5} \\
    V_b = -\dfrac{8}{5}\\
    V_c = -\dfrac{1}{5} \\
    V_d = -\dfrac{1}{5}
    \end{cases*}}
    \end{align*}

}

\qitem{
	Draw the superposition circuit for $V_3$ and write the equation that characterizes this circuit.
}

\ans{

    \begin{center}
    \begin{circuitikz}[line width=0.5pt]
    
    	\draw[color=brown](0,0) %Node A
    	to[short, l^=$0V$] ++(0,0)
    	to[short] ++(1,0)
    	to[short] ++(-1,0)
    	to[short] ++(0,1)
    	to[short] ++(0,-1)
    	to[short] node[ground] {} ++(0,-1);
    
    \draw[color=green](0,2) %Node B
    	to[short] ++(0,1)
    	to[short, l^=$V_1$] ++(0,0)
    	to[short] ++(1,0);
    
    \draw[color=red](2,3) %Node C
    	to[short] ++(2,0)
    	to[short] ++(-1,0)
    	to node[above]{$V_a$} ++(0,0)
    	to[short] ++(0,-1);
    
    \draw[color=blue](5,3)
    	to[short] ++(1,0)
    	to[short, l_=$V_b$] ++(0,0)
    	to[short] ++(0,-1);
    
    \draw[color=magenta](6,1)
    	to[short] ++(0,-1)
    	to[short, l_=$V_c$] ++(0,0)
    	to[short] ++(-1,0);
    
    \draw[color=orange](2,0)
    	to[short] ++(2,0)
    	to[short] ++(-1,0)
    	to node[below] {$V_d$} ++(0,0)
    	to[short] ++(0,1);
    
    \draw(1,0) to[R=$R_1$] ++(1,0);% v=$ $, l_=$\stackrel{i_1}{\longrightarrow}{R_1}$] ++(1,0);
    \draw(1,3) to[R=$R_2$] ++(1,0); %v=$ $, l_=$\stackrel{i_2}{\longrightarrow}{R_2}$] ++(1,0);
    \draw(4,3) to[short,l=$V_a \eq V_b$] ++(1,0);
    \draw(4,0) to[V,l=$V_3$] ++(1,0);
    \draw(3,2) to[R=$R_3$] ++(0,-1);%, v=$ $, l_=$\stackrel{i_3}{\downarrow}{R_3}$] ++(0,-1);
    \draw(6,2) to[R=$R_4$] ++(0,-1);%, v=$ $, l_=$\stackrel{i_4}{\downarrow}{R_4}$] ++(0,-1);
    \draw(0,1) to[short, l=$V_1 \eq 0$] ++(0,1);
    
    \end{circuitikz}
    \end{center}
    
    \begin{align*}
    \begin{cases*}
    \dfrac{V_a-V_1}{R_2} + \dfrac{V_a - V_d}{R_3} + \dfrac{V_b - V_c}{R_4} = 0\\
    \dfrac{V_d}{R_1} + \dfrac{V_d - V_a}{R_3} + \dfrac{V_c - V_b}{R_4} = 0 \\
    V_d = V_c + V_3 = V_c + 4\\
    V_a = V_b\\
    V_1 = 0
    \end{cases*}
    &\implies
    \begin{cases*}
    \dfrac{V_a}{2} + \dfrac{V_a - V_d}{4} + \dfrac{V_b - V_c}{4} = 0 \\
    V_d + \dfrac{V_d - V_a}{4} + \dfrac{V_c - V_b}{4} = 0 \\
    V_d = V_c + 4
    V_a = V_b
    \end{cases*}
    \end{align*}
    We find that: $\boxed{V_a = -\frac{4}{5}\text{V}, V_b = -\frac{4}{5}\text{V}, V_c = -\frac{18}{5}\text{V}, V_d = \frac{2}{5}\text{V}}$
}

\qitem{
    Find the unknown potentials as you did in in part (d), but now do so using the superposition theorem.
}

\ans{
    By the superposition theorem, the actual potential values at nodes $V_a$ through $V_d$ are the linear sums of their corresponding potentials in the simpler circuits. Notice that we kept the ground in the same place for each of the simpler circuits! While the location of the ground does not matter, \textit{keeping it consistent does}. We add up the potentials and find that:
    \begin{align*}
        V_a &= \dfrac{3}{5} + \dfrac{2}{5} - \dfrac{4}{5} = \dfrac{1}{5}\\
        V_b &= \dfrac{3}{5} - \dfrac{8}{5} - \dfrac{4}{5} = -\dfrac{9}{5} \\
        V_c &= \dfrac{1}{5} - \dfrac{1}{5} - \dfrac{18}{5} = -\dfrac{18}{5} \\
        V_d &= \dfrac{1}{5} - \dfrac{1}{5} + \dfrac{2}{5} = \dfrac{2}{5}
    \end{align*}
    Wow! The same potentials as we found using supernode/nodal analysis!
}

\end{enumerate}

	




% Author: Mudit Gupta
% Email: mudit+csm16a@berkeley.edu

\qns{Null space drill}

\sol{Prereq: Introduction to nullspaces. A mini-lecture gets you ready.  \\
Description: First a proof about nullspaces, and then lots of mechanical practice on nullspaces.}

In this question, we explore intuition about null spaces and a recipe to compute them. Recall that the nullspace of a matrix $\mathbf{M}$ is the set of all vectors, $\vec{x}$ such that $\mathbf{M}\vec{x} = \vec{0}$.

\begin{enumerate}

\qitem{First, we begin by proving that a null $\text{space}$ is indeed a subspace. Show that any nullspace of a matrix $\mathbf{M}$ with $n$ rows and $n$ columns is a subspace. \\
Steps for reference:
\begin{enumerate}
 \item claim subset of X
 \item claim X is known vector space
 \item closures and 0 
 \begin{enumerate}
 \item closure under addition
 \item closure under scalar multiplication
 \item existence of the zero element.
 \end{enumerate}	
 \end{enumerate} }

\ans{
	\begin{enumerate}
	\item A nullspace of a matrix with $n$ rows and $n$ columns must contain vectors of $n$ elements. These vectors clearly form a subset of $\mathbb{R}^n$. 
	\item $\mathbb{R}^n$ is a known vector space. 
	\item Closures and 0
	\begin{enumerate} \item Consider two elements, $\vec{x_1}$ and $\vec{x_2}$ in the nullspace of $\mathbf{M}$. By definition, we know that $\mathbf{M}\vec{x_1} = \vec{0}$ and $\mathbf{M}\vec{x_2} = \vec{0}$. 
	Now consider the vector $\vec{x_1} + \vec{x_2}$. Is $\mathbf{M}(\vec{x_1} + \vec{x_2}) \stackrel{?}{=} \vec{0}$. $\mathbf{M}\vec{x_1} + \mathbf{M}\vec{x_2} = \vec{0} + \vec{0} = \vec{0}$. Done. 
	\item Consider another element, $\vec{x_3}$ in the nullspace of $\mathbf{M}$. Consider a scalar $a \in \mathbb{R}$. Is $a\vec{x_2}$ in the nullspace of $\mathbf{M}$, i.e., is $\mathbf{M}(a\vec{x_3}) \stackrel{?}{=} \vec{0}$. Yes, because $a\mathbf{M}\vec{x_3} = 0$ since $\vec{x_3}$ is in the nullspace. 
	\item Is $\vec{0}$ in the nullspace of $\mathbf{M}?$ Yes, because $\mathbf{M}\vec{0} = \vec{0}$. 
\end{enumerate}
\end{enumerate}
	Therefore, a nullspace is indeed a subspace.
}

\qitem{Now we will explore a recipe to compute null spaces. Let's start with some 3x3 matrices. \\

$$\mathbf{A} = \begin{bmatrix} 1 & -3 & 1 \\ 2 & -8 & 8 \\ -6 & 3 & -15 \end{bmatrix}$$
$\mathbf{A^\prime}$ is the row reduced matrix $\mathbf{A}$.
$$\mathbf{A^{\prime}} = \begin{bmatrix} 1 & -3 & 1 \\ 0 & -1 & 3 \\ 0 & 0 & -18 \end{bmatrix}$$
Compute the nullspace of $\mathbf{A}$.
 }

\ans{
Since the row reduced matrix $\mathbf{A^\prime}$ has a pivot in every column, the matrix has a trivial nullspace. The nullspace is the vector $\vec{0}$.

Let's look at this in more depth, however. Remember that a nullspace is the set of vectors such that $\mathbf{A}{\vec{x}} = \vec{0}.$ Let's solve this as linear equations. 

$$\begin{bmatrix} 1 & -3 & 1 \\ 2 & -8 & 8 \\ -6 & 3 & -15 \end{bmatrix} \begin{bmatrix} x_1 \\ x_2 \\ x_3 \end{bmatrix} = \begin{bmatrix} 0 \\ 0 \\ 0 \end{bmatrix}$$

To solve this, we row reduce. This results in

$$\begin{bmatrix} 1 & -3 & 1 \\ 0 & -1 & 3 \\ 0 & 0 & -18 \end{bmatrix} \begin{bmatrix} x_1 \\ x_2 \\ x_3 \end{bmatrix} = \begin{bmatrix} 0 \\ 0 \\ 0 \end{bmatrix}$$

Let's convert this back to linear equations:

$$x_1 - 3x_2 + x_3 = 0$$ 
$$ -x_2 + x_3 = 0$$
$$ -18x_3 = 0$$

The third equation is only satisfied by $x_3 = 0$. The second equation implies that $x_2 = x_3 = 0$. And finally, the first equation is also only satisfied by $x_1 = 0$. Therefore, $\begin{bmatrix} 0 \\ 0 \\ 0 \end{bmatrix}$ is the only vector which satisfies these equations

}

\qitem{
	Consider another matrix $$\mathbf{B} = \begin{bmatrix} 1 & -1 & 2 \\ 4 & 4 & -2 \\ -2 & 2 & -4 \end{bmatrix}$$ $\mathbf{B^\prime}$ is row reduced $\mathbf{B}$. $$\mathbf{B^\prime} = \begin{bmatrix} 1 & -1 & 2 \\ 0 & 8 & -10 \\ 0 & 0 & 0 \end{bmatrix}$$

	What is the null space of $\mathbf{B}$? What is the dimension of the row space of $\mathbf{B}$?
}

\ans{
	Think of this as linear equations once again. Let the first column correspond to $x$, the second to $y$ and the third to $z$. In equation form, the row reduced matrix becomes 

	\begin{equation} \label{nullpractice:1} x - y + 2z = 0 \end{equation}
	\begin{equation} \label{nullpractice:2} 8y - 10z = 0 \end{equation}
	\begin{equation} \label{nullpractice:3} 0x + 0y + 0z = 0 \end{equation}

	Equation \ref{nullpractice:3} gives us no information -- it is always true. So we ignore it. \\

	Equation \ref{nullpractice:2} says that $4y = 5z$. Let's set $z = t$ (let $z$ be a free variable that can take on any value). Then $y = \frac{5}{4} t$. \\

	Equation \ref{nullpractice:1} is then $x - \frac{5}{4}t + 2t = 0 \implies x = \frac{-3}{4}t$. The nullspace is then all vectors of the form $t\begin{bmatrix} \frac{-3}{4} \\ \frac{5}{4} \\ 1 \end{bmatrix}$, where $t$ is any real number. Another way to say this is that the nullspace is spanned by the vector 

	\begin{equation} \label{nullpractice:4} \begin{bmatrix} \frac{-3}{4} \\ \frac{5}{4} \\ 1 \end{bmatrix} \end{equation} 

	The dimension of the nullspace, i.e., the minimum number of vectors required to span it is $1$. 

	From the rank-nullity theorem, we know that Dim(Rowspace($\mathbf{B}$)) + Dim(Nullspace($\mathbf{B}$)) = Number of columns in $\mathbf{B}$. Therefore, the dimension of the rowspace of $\mathbf{B}$ is 2. 

}

\sol{Mentors: State the rank nullity theorem without proof. For any matrix, $\mathbf{A}$, Rank($\mathbf{A}$) + Nullity($\mathbf{A}$) = number of columns in $\mathbf{A}$. Rank($\mathbf{A}$) = dim(colspace($\mathbf{A}$)) = dim(rowspace($\mathbf{A}$)). Nullity($\mathbf{A}$) = dim(nullspace($\mathbf{A}$))}

\qitem{In the previous part, we chose one of the variables and set it to be a free variable. Can we choose any variable as our free variable?}

\ans{Let's investigate this question by choosing each variable as a free variable. We know $z$ works from the solution to the previous part. \\

Let's consider $y$. If we set $y = t$ instead, then we get, from Equation (\ref{nullpractice:2}) $5z = 4t \implies z = \frac{4}{5}t$. \\

Equation (\ref{nullpractice:1}) then gives us $x - t + 2\frac{4}{5}t = 0 \implies x - \frac{5t}{5} + \frac{8t}{5} = 0 \implies x = \frac{-3t}{5}$. \\

The nullspace is then spanned by the vector $\begin{bmatrix} \frac{-3}{5} \\ 1 \\ \frac{4}{5} \end{bmatrix}$ \\

Note that this vector is $\frac{4}{5}$ times the vector we found in (\ref{nullpractice:4}). \\ 

}

\sol{At this point, stress that the choice of free variable doesn't change the null space. Since subspaces are closed under scalar multiplication, the fact that this vector is a multiple of the previous shouldn't be a surprise to students. If it is, explain why this is the case.}

\ans{

Now, let's see what happens if we set $x=t$ instead. We can't use Equation (\ref{nullpractice:2}) yet, so let's try using Equation (\ref{nullpractice:1}). $t - y + 2z = 0$. Now what? How do we find the value for $y$ or $z$ in terms of $t$? ...We can't. So $x$ does not work...

}

\qitem{How can we know which variables can be used as free variables?}

\ans{Pick your free variables are by looking at columns with no pivots. Although, sometimes, other variables might work (like $y$ above), the variables with no pivots will always work!}

\qitem{
	Now consider another matrix, $\mathbf{C} = \begin{bmatrix} 1 & -2 & -6 & 12 \\ 2 & 4 & 12 & -17 \\ 1 & -4 & -12 & 22 \end{bmatrix}$
	Without doing any math, will this matrix have a trivial nullspace, i.e. consisting of only $\vec{0}$?
}

\ans{No! A 3x4 matrix can simply not have 4 pivots. So at least one of the variables will need to be free!}

\qitem{
	Consider another matrix, $\mathbf{D} = \begin{bmatrix} 1 & -2 & -6 & 12 \\ 0 & -2 & -6 & 10 \\ 0 & 0 & 0 & -1 \end{bmatrix}$. Find vector(s) that span the nullspace.
}

\ans{
	In terms of equations, let the variables for cols 1-4 be $a$ to $d$ respectively. Column 3 does not have a pivot. So $c$ is free. Let $c = t$. At this point, we should feel comfortable reading the matrix as its equations without explicitly writing the equations! \\

	Row 3 of the matrix says that $-d = 0$, or that $d = 0$. \\

	Row 2 says that $-2b -6c + 10d = 0 \implies -2b -6t = 0 \implies b=-3t$. \\

	Row 1 says that $a - 2b -6c + 12d = 0 \implies a + 6t -6t = 0 \implies a=0$. \\

	The vector that spans this nullspace is $\begin{bmatrix} 0 \\ -3 \\ 1 \\ 0\end{bmatrix}$

	}
\sol{
	Students could be confused about 'pivots'. Column 3 doesn't have a pivot because it has a 0 in the place of the 'diagonal' instead. Also, ask them which column doesn't have a pivot in this case: $\begin{bmatrix} 1 & -2 & -6 & 12 \\ 0 & -2 & -6 & 10 \\ 0 & 0 & -1 & 9 \end{bmatrix}$ (note that the last row is different.) Basically, column 4 doesn't a pivot, because that pivot would have been in the 4th row which doesn't exist. \textit{Make sure these concepts about pivots settle in}.

}
\qitem{Consider one final matrix, $\mathbf{E} = \begin{bmatrix} 1 & -2 & -6 & 12 \\ 0 & -2 & -6 & 10 \\ 0 & 0 & 0 & 0\end{bmatrix}$. What are the vector(s) that span this nullspace?}

\ans{
	Again, let the variables for the columns be $a$ to $d$ respectively. Columns 3 and 4 don't have pivots. So let's set both of them to be free! \\

	Let $c = t, d = s$. \\

	Row 2 says $-2b -6c + 10d = 0 \implies -2b = 6t - 10s \implies b = -3t + 5s$. \\

	Row 1 says $a -2b -6c + 10d = 0 \implies a = 0$. \\

	The general form of vectors in the nullspace is then $\begin{bmatrix} 0 \\ -3t + 5s \\ t \\ s \end{bmatrix}$. This needs to be rewritten by splitting the free variables $s\begin{bmatrix} 0 \\ 5 \\ 0 \\ 1 \end{bmatrix} + t\begin{bmatrix} 0 \\ -3 \\ 1 \\ 0\end{bmatrix}$. \\

	Finally we conclude that the vectors that span the nullspace are $\begin{bmatrix} 0 \\ 5 \\ 0 \\ 1 \end{bmatrix}$ and $\begin{bmatrix} 0 \\ -3 \\ 1 \\ 0 \end{bmatrix}$. \\

	Observation: notice that the number of free variables = number of columns without pivot = number of vectors required to span the nullspace = dimension of the nullspace!
}




\end{enumerate}

\input{../resistor_equivalences/resistor_equivalence_recipe} %Note: This one has 2 problems back to back

% Author: Mudit Gupta
% Email: mudit+csm16a@berkeley.edu

\qns{Explore Subspace}

\sol{Prereqs: What are vector spaces and subspaces? \\
Description: Explains how to read set notation, tries to make students really realize that the notation means a set of vectors, and that a subspace is also a set of vectors. And what a subspace intuitively means.}

\begin{enumerate}

\qitem{
	Consider the set $W = \{\begin{bmatrix} a_1 \\ a_2 \\ a_3 \end{bmatrix}, a_1, a_2, a_3 \in \mathbb{R}: a_1 + 2a_2 - 3a_3 = 0\}$. Is $\begin{bmatrix} 1 \\ 2 \\ 3 \end{bmatrix}$ an element of the set $W$?
}

\ans{
	The way set notation works is that for an element to be a part of the set, $a_1 + 2a_2 - 3a_3$ must be satisfied. We can plug in numbers for $a_1, a_2, a_3$ from the vector and see if the equation is satisfied. $1 + 4 - 9 \neq 0$, so this element is not a member of the set.
}

\qitem{Write any 3 elements from this set}

\sol{The purpose here is really to make sure the students realize that $W$ is indeed a set. It has infinite number of elements, but it is still a set. Let the students come up with whatever they want. Ultimately though, it will be super helpful if the final 3 elements you get are the same as the ones in the answer below. We will be using these again! So verify a couple of the elements that the students put forth, but ultimately write these on the board.}

\ans{There are many possibilities. For instance, we could set the values of $a_2$ and $a_3$ to be $0$ and then see what value of $a_1$ works. $\vec{v}_1 = \begin{bmatrix} 0 \\ 0 \\ 0 \end{bmatrix}$ is in the set. \\

Another element can be obtained by setting the values of $a_1$ and $a_2$ to be $1$, and then we get $1 + 2 - 3a_3 = 0$, or $a_3 = 1$. Therefore $\vec{v}_2 = \begin{bmatrix} 1 \\ 1 \\ 1 \end{bmatrix}$ is also in the set. \\

Another element can be obtained by setting the values of $a_1 = 3$ and $a_3 = 2$ and then we get $3 + 2a_2 - 6 = 0 \implies a_2 = \frac{3}{2}$. Therefore $\vec{v}_3 = \begin{bmatrix} 3 \\ \frac{3}{2} \\ 2 \end{bmatrix}$ is in the set too.}

\qitem{Is the set $W$ a subspace?}

\ans{
	\textbf{Step 1: Claim that $W$ is a sub$\textit{set}$ of, say, $X$}. \\
	$W$ is clearly a subset of $\mathbb{R}^3$. This can be seen because the elements of $W$ contain 3 elements, but $W$ is not equal to $\mathbb{R}^3$ since some elements from $\mathbb{R}^3$ are not in $W$. \\ 

	\textbf{Step 2: Claim that $X$ is a vector space} \\
	$\mathbb{R}^3$ is a vector space that we have seen in lecture. \\ 

	\textbf{Step 3: If $X$ is a known vector space, and $W$ is a subset of $X$, only 3 axioms must be proven.} \\ \\
	\textbf{a: Prove closure under addition} \\
	Consider two arbitrary elements from the set $\begin{bmatrix} b_1 \\ b_2 \\ b_3 \end{bmatrix}$ and $\begin{bmatrix} c_1 \\ c_2 \\ c_3 \end{bmatrix}$. Since these elements are a part of the set, it is true that $b_1 + 2b_2 - 3b_3 = 0$ and that $c_1 + 2c_2 - 3c_3 = 0$. \\
	Consider the sum of these elements. $\begin{bmatrix} b_1 + c_1 \\ b_2 + c_2 \\ b_3 + c_3 \end{bmatrix}$. Is this element a part of the set too? In other words, is $(b_1+c_1) + 2(b_2+c_2) - 3(b_3 + c_3) = 0$? \\
	$$(b_1+c_1) + 2(b_2+c_2) - 3(b_3 + c_3) \stackrel{?}{=} 0$$
	$$\implies (b_1 + 2b_2 - 3b_3) + (c_1 + 2c_2 - 3c_3) \stackrel{?}{=} 0$$
	Clearly, the left hand side equals the right hand side. Therefore, 
	$$0 + 0 = 0$$.
	Thus, the element $\begin{bmatrix} b_1 + c_1 \\ b_2 + c_2 \\ b_3 + c_3 \end{bmatrix}$ is a part of the set and we have proven closure under addition. \\ 

	\textbf{b: Prove closure under scalar multiplication} \\
	Consider an arbitrarily element from the set $\begin{bmatrix} d_1 \\ d_2 \\ d_3 \end{bmatrix}$. This means that $d_1 + 2d_2 - 3d_3 = 0$ is true. \\
	Consider some scalar $s$. Is $s\begin{bmatrix} d_1 \\ d_2 \\ d_3 \end{bmatrix}$ in the set $W$? I.e., is $\begin{bmatrix} sd_1 \\ sd_2 \\ sd_3 \end{bmatrix}$ in the set $W$? I.e. is $sd_1 + 2sd_2 - 3sd_3 = 0$?
	$$sd_1 + 2sd_2 - 3sd_3 \stackrel{?}{=} 0$$
	$$\implies s \cdot (d_1 + 2d_2 - 3d_3) \stackrel{?}{=} 0$$
	Indeed the left hand side of the equation equals the right hand side, i.e., $$s\cdot 0 = 0$$

	Therefore, $\begin{bmatrix} sd_1 \\ sd_2 \\ sd_3 \end{bmatrix}$ is in the set $W$ and the set $W$ is closed under scalar multiplication. \\ 

	\textbf{c: Prove existence of 0 element} \\
	We need to check whether $\vec{0} = \begin{bmatrix} 0 \\ 0 \\ 0 \end{bmatrix}$ exists in the set. This is easy to check because we just need to check whether $0 + 2 \cdot 0 - 3 \cdot 0 = 0$, which it is. So the 0 element exists in the set. \\ 

	Therefore, the set $W$ is a subspace of $\mathbb{R}^3$. 
}

\qitem{How can we now quickly find more elements of this set?}

\ans{
	Since we know the set is closed under scalar multiplication and under addition, we can easily find more elements. Previously, we found $\vec{v}_1 = \begin{bmatrix} 1 \\ 1 \\ 1 \end{bmatrix}$ to be an element. Now we know that any multiple of this is in the set too! \\
	We also found that $\vec{v}_2 = \begin{bmatrix} 3 \\ \frac{3}{2} \\ 2 \end{bmatrix}$ was in the set. Now we can add the 2 elements we found, and $\begin{bmatrix} 4 \\ \frac{5}{2} \\ 3 \end{bmatrix}$ is also in the set. In fact, any $s\vec{v}_1 + r\vec{v}_2$, where $s, r \in \mathbb{R}$ are in the set! 
}



\end{enumerate}

% Author: Mudit Gupta
% Email: mudit+csm16a@berkeley.edu

\qns{Eigenvalues everywhere}

\sol{Prereq: All of linear algebra basically, including page rank et al.  \\
Description: Meant to be an intuition problem on eigenvalues and eigenvectors.}

In this problem, when asked for eigenvectors, you may simply state that the eigenvector comes from a set. For instance, you could state that any $\vec{x} \in \text{Colspace}(\mathbf{A})$ is an eigenvector. Also, note that when asked to find eigenvalues, only consider real eigenvalues for this problem. 
\begin{enumerate}
\qitem {
	What is the one eigenvalue and eigenvector of the matrix that you can see without solving any equations? $$\mathbf{A} = \begin{bmatrix} 1 & 2 \\ 0 & 0 \end{bmatrix}$$ 
}


\ans{
	Since this matrix is clearly not-invertible, it must have an eigenvalue $0$. 

	$$\mathbf{A}\vec{x} = \lambda\vec{x}$$
	$$\mathbf{A}\vec{x} = 0\vec{x}$$
	$$\mathbf{A}\vec{x} = \vec{0}$$
	This equation is precisely the equation for computing the nullspace of $\mathbf{A}$. Therefore, any $\vec{x} \in \text{Nullspace}(\mathbf{A})$ works.
}

\sol {
	The point of this problem is not to find the eigenvalues mechanically, but instead use properties of the matrix that you can eyeball to figure out some eigenvalues and eigenvectors. Don't spend time mechancially computing the eigenvalues.
}

\qitem {
	What are the eigenvalues and eigenvectors of the matrix $$\mathbf{B} = \begin{bmatrix} 3 & 0 & 0 \\ 0 & 3 & 0 \\ 0 & 0 & 3\end{bmatrix}$$

}

\ans{
	This is a scaling matrix. It scales any vector by a factor of $3$. What this means is that any vector $\vec{x} \in \mathbb{R}^3$ when post-multiplied by $\mathbf{A}$ will output $3\vec{x}$. This matrix has only one eigenvalue, $\lambda = 3$ and any $\vec{x} \in \mathbb{R}^3$ is an eigenvector. 
}

\qitem{
	What are the eigenvalues of $$\mathbf{C} = \begin{bmatrix} 2 & 0 \\ 3 & 4 \\ 1 & 3 \end{bmatrix}?$$
}

\ans{
	This is a trick question. Eigenvalues are defined only for square matrices.
}

\qitem{
	Consider a matrix that rotates a vector in $\mathbb{R}^2$ by $45^\circ$ counterclockwise. For instance, it rotates any vector along the x-axis to orient towards the $y=x$ line. Find its eigenvalues and corresponding eigenvectors. This matrix is given as $$\mathbf{D} = \begin{bmatrix} \cos45 & -\sin45 \\ \sin45 & \cos45 \end{bmatrix} = \dfrac{\sqrt{2}}{2} \begin{bmatrix} 1 & -1 \\ 1 & 1 \end{bmatrix}$$
}
\sol{Please draw a picture to show what the matrix does to a vector. Also remember we are only considering real eigenvalues, as written in the prompt of the problem.}

\ans{
	Remember that the equation $\mathbf{A}\vec{x} = \lambda\vec{x}$ geometrically means that for the matrix $\mathbf{A}$, there exist some special vectors $\vec{x}$ that are merely scaled by $\lambda$ when post-multiplied by $\mathbf{A}$. For a matrix that takes a vector and rotates it by $45^\circ$, there are no real-valued vectors that it can simply scale. This means that there are no real eigenvalues for this matrix either. 
}

\qitem{
	What are the eigenvalues of the following matrix? $$\mathbf{E} = \begin{bmatrix} 1 & \frac{1}{2} & \frac{1}{3} \\ 0 & \frac{1}{2} & \frac{1}{3} \\ 0 & 0 & \frac{1}{3}\end{bmatrix}$$
}

\ans{
	Remember that for upper triangular matrices, the eigenvalues can be read from the diagonal. $1, \frac{1}{2}, \frac{1}{3}$ are the three eigenvalues.
}

\qitem{
	Can you find an eigenvalue of the following matrix without solving any equations?
	$$\mathbf{F} = \begin{bmatrix} 1 & 0 & 0 \\ \frac{1}{3} &\frac{1}{3} &\frac{1}{3} \\\frac{1}{2} & \frac{1}{4} & \frac{1}{4}  \end{bmatrix}$$

}
\ans{
	This is a matrix whose rows sum to 1, therefore, it has an eigenvalue 1. 

	This is proven by letting $\vec{x} = \begin{bmatrix} 1 \\ 1 \\ 1 \end{bmatrix}$ be a potential eigenvector of the matrix $\mathbf{F}$.
	Looking at the column view of matrix-vector multiplication -- 
	$$\mathbf{F}\begin{bmatrix} 1 \\ 1 \\ 1 \end{bmatrix}
	 = 1 \cdot \begin{bmatrix} 1 \\ \frac{1}{3} \\ \frac{1}{2} \end{bmatrix} + 1\cdot\begin{bmatrix} 0 \\ \frac{1}{3} \\ \frac{1}{4} \end{bmatrix} + 1\cdot\begin{bmatrix} 0 \\ \frac{1}{3} \\ \frac{1}{4}\end{bmatrix}$$ 
	 $$\mathbf{F}\vec{x} = 1\cdot\vec{x}$$ since the rows sum to one.

	 Therefore, 1 is an eigenvalue with corresponding eigenvector $\begin{bmatrix} 1 \\ 1 \\ 1 \end{bmatrix}$
}

\sol{
	Make sure students see why this works generally. Essentially $\mathbf{A}\begin{bmatrix} 1 \\ 1 \\ 1 \end{bmatrix} = 1\cdot\vec{v}_1 + 1\cdot\vec{v}_2 + 1\cdot\vec{v}_3 = \begin{bmatrix} 1 \\ 1 \\ 1\end{bmatrix}$, where $\vec{v}_i$ are the columns of $\mathbf{A}$, and the sum equals $\begin{bmatrix} 1 \\ 1 \\ 1\end{bmatrix}$ because each row sums to one. 
}

\qitem{Show that a matrix and its transpose have the same eigenvalues

Hint: The determinant of a matrix is the same as the determinant of its transpose}



\ans{
	For any matrix $\mathbf{M}$, 
	$$det(\mathbf{M}) = det(\mathbf{M}^T)$$

	Eigenvalues are found by solving the equation $det(\mathbf{M} - \lambda\mathbf{I}) = 0$. \\

	Note that $(\mathbf{M} - \lambda\mathbf{I})^T = \mathbf{M}^T - \lambda\mathbf{I}^T = \mathbf{M}^T - \lambda\mathbf{I}$. \\

	Let $\mathbf{M} - \lambda\mathbf{I} = \mathbf{G}$. 
	$$det(\mathbf{G}) = det(\mathbf{G}^T)$$
	$$det(\mathbf{M} - \lambda\mathbf{I}) = det(\mathbf{M}^T - \lambda\mathbf{I})$$
	If we set the left hand side to 0 to solve for the lambdas, we also extract the lambdas corresponding to the right hand side. Therefore, $\mathbf{M}$ and its transpose have the same eigenvalues.
}

\qitem{
	Consider a matrix whose columns sum to one. What is one possible eigenvalue of this matrix?
}

\ans{
	We showed that for any matrix like $\mathbf{F}$ whose rows sum to 1, one eigenvalue is 1. We also showed that a matrix and its transpose have the same eigenvalues. Consider $\mathbf{F}^T$. It has columns summing to 1. Therefore, 1 is an eigenvalue of $\mathbf{F}^T$ too, and by extension of all matrices whose columns sum to one. 
}
\end{enumerate}

% Author: Mudit Gupta
% Email: mudit+csm16a@berkeley.edu

\qns{Null space drill}

\sol{Prereq: Introduction to nullspaces. A mini-lecture gets you ready.  \\
Description: First a proof about nullspaces, and then lots of mechanical practice on nullspaces.}

In this question, we explore intuition about null spaces and a recipe to compute them. Recall that the nullspace of a matrix $\mathbf{M}$ is the set of all vectors, $\vec{x}$ such that $\mathbf{M}\vec{x} = \vec{0}$.

\begin{enumerate}

\qitem{First, we begin by proving that a null $\text{space}$ is indeed a subspace. Show that any nullspace of a matrix $\mathbf{M}$ with $n$ rows and $n$ columns is a subspace. \\
Steps for reference:
\begin{enumerate}
 \item claim subset of X
 \item claim X is known vector space
 \item closures and 0 
 \begin{enumerate}
 \item closure under addition
 \item closure under scalar multiplication
 \item existence of the zero element.
 \end{enumerate}	
 \end{enumerate} }

\ans{
	\begin{enumerate}
	\item A nullspace of a matrix with $n$ rows and $n$ columns must contain vectors of $n$ elements. These vectors clearly form a subset of $\mathbb{R}^n$. 
	\item $\mathbb{R}^n$ is a known vector space. 
	\item Closures and 0
	\begin{enumerate} \item Consider two elements, $\vec{x_1}$ and $\vec{x_2}$ in the nullspace of $\mathbf{M}$. By definition, we know that $\mathbf{M}\vec{x_1} = \vec{0}$ and $\mathbf{M}\vec{x_2} = \vec{0}$. 
	Now consider the vector $\vec{x_1} + \vec{x_2}$. Is $\mathbf{M}(\vec{x_1} + \vec{x_2}) \stackrel{?}{=} \vec{0}$. $\mathbf{M}\vec{x_1} + \mathbf{M}\vec{x_2} = \vec{0} + \vec{0} = \vec{0}$. Done. 
	\item Consider another element, $\vec{x_3}$ in the nullspace of $\mathbf{M}$. Consider a scalar $a \in \mathbb{R}$. Is $a\vec{x_2}$ in the nullspace of $\mathbf{M}$, i.e., is $\mathbf{M}(a\vec{x_3}) \stackrel{?}{=} \vec{0}$. Yes, because $a\mathbf{M}\vec{x_3} = 0$ since $\vec{x_3}$ is in the nullspace. 
	\item Is $\vec{0}$ in the nullspace of $\mathbf{M}?$ Yes, because $\mathbf{M}\vec{0} = \vec{0}$. 
\end{enumerate}
\end{enumerate}
	Therefore, a nullspace is indeed a subspace.
}

\qitem{Now we will explore a recipe to compute null spaces. Let's start with some 3x3 matrices. \\

$$\mathbf{A} = \begin{bmatrix} 1 & -3 & 1 \\ 2 & -8 & 8 \\ -6 & 3 & -15 \end{bmatrix}$$
$\mathbf{A^\prime}$ is the row reduced matrix $\mathbf{A}$.
$$\mathbf{A^{\prime}} = \begin{bmatrix} 1 & -3 & 1 \\ 0 & -1 & 3 \\ 0 & 0 & -18 \end{bmatrix}$$
Compute the nullspace of $\mathbf{A}$.
 }

\ans{
Since the row reduced matrix $\mathbf{A^\prime}$ has a pivot in every column, the matrix has a trivial nullspace. The nullspace is the vector $\vec{0}$.

Let's look at this in more depth, however. Remember that a nullspace is the set of vectors such that $\mathbf{A}{\vec{x}} = \vec{0}.$ Let's solve this as linear equations. 

$$\begin{bmatrix} 1 & -3 & 1 \\ 2 & -8 & 8 \\ -6 & 3 & -15 \end{bmatrix} \begin{bmatrix} x_1 \\ x_2 \\ x_3 \end{bmatrix} = \begin{bmatrix} 0 \\ 0 \\ 0 \end{bmatrix}$$

To solve this, we row reduce. This results in

$$\begin{bmatrix} 1 & -3 & 1 \\ 0 & -1 & 3 \\ 0 & 0 & -18 \end{bmatrix} \begin{bmatrix} x_1 \\ x_2 \\ x_3 \end{bmatrix} = \begin{bmatrix} 0 \\ 0 \\ 0 \end{bmatrix}$$

Let's convert this back to linear equations:

$$x_1 - 3x_2 + x_3 = 0$$ 
$$ -x_2 + x_3 = 0$$
$$ -18x_3 = 0$$

The third equation is only satisfied by $x_3 = 0$. The second equation implies that $x_2 = x_3 = 0$. And finally, the first equation is also only satisfied by $x_1 = 0$. Therefore, $\begin{bmatrix} 0 \\ 0 \\ 0 \end{bmatrix}$ is the only vector which satisfies these equations

}

\qitem{
	Consider another matrix $$\mathbf{B} = \begin{bmatrix} 1 & -1 & 2 \\ 4 & 4 & -2 \\ -2 & 2 & -4 \end{bmatrix}$$ $\mathbf{B^\prime}$ is row reduced $\mathbf{B}$. $$\mathbf{B^\prime} = \begin{bmatrix} 1 & -1 & 2 \\ 0 & 8 & -10 \\ 0 & 0 & 0 \end{bmatrix}$$

	What is the null space of $\mathbf{B}$? What is the dimension of the row space of $\mathbf{B}$?
}

\ans{
	Think of this as linear equations once again. Let the first column correspond to $x$, the second to $y$ and the third to $z$. In equation form, the row reduced matrix becomes 

	\begin{equation} \label{nullpractice:1} x - y + 2z = 0 \end{equation}
	\begin{equation} \label{nullpractice:2} 8y - 10z = 0 \end{equation}
	\begin{equation} \label{nullpractice:3} 0x + 0y + 0z = 0 \end{equation}

	Equation \ref{nullpractice:3} gives us no information -- it is always true. So we ignore it. \\

	Equation \ref{nullpractice:2} says that $4y = 5z$. Let's set $z = t$ (let $z$ be a free variable that can take on any value). Then $y = \frac{5}{4} t$. \\

	Equation \ref{nullpractice:1} is then $x - \frac{5}{4}t + 2t = 0 \implies x = \frac{-3}{4}t$. The nullspace is then all vectors of the form $t\begin{bmatrix} \frac{-3}{4} \\ \frac{5}{4} \\ 1 \end{bmatrix}$, where $t$ is any real number. Another way to say this is that the nullspace is spanned by the vector 

	\begin{equation} \label{nullpractice:4} \begin{bmatrix} \frac{-3}{4} \\ \frac{5}{4} \\ 1 \end{bmatrix} \end{equation} 

	The dimension of the nullspace, i.e., the minimum number of vectors required to span it is $1$. 

	From the rank-nullity theorem, we know that Dim(Rowspace($\mathbf{B}$)) + Dim(Nullspace($\mathbf{B}$)) = Number of columns in $\mathbf{B}$. Therefore, the dimension of the rowspace of $\mathbf{B}$ is 2. 

}

\sol{Mentors: State the rank nullity theorem without proof. For any matrix, $\mathbf{A}$, Rank($\mathbf{A}$) + Nullity($\mathbf{A}$) = number of columns in $\mathbf{A}$. Rank($\mathbf{A}$) = dim(colspace($\mathbf{A}$)) = dim(rowspace($\mathbf{A}$)). Nullity($\mathbf{A}$) = dim(nullspace($\mathbf{A}$))}

\qitem{In the previous part, we chose one of the variables and set it to be a free variable. Can we choose any variable as our free variable?}

\ans{Let's investigate this question by choosing each variable as a free variable. We know $z$ works from the solution to the previous part. \\

Let's consider $y$. If we set $y = t$ instead, then we get, from Equation (\ref{nullpractice:2}) $5z = 4t \implies z = \frac{4}{5}t$. \\

Equation (\ref{nullpractice:1}) then gives us $x - t + 2\frac{4}{5}t = 0 \implies x - \frac{5t}{5} + \frac{8t}{5} = 0 \implies x = \frac{-3t}{5}$. \\

The nullspace is then spanned by the vector $\begin{bmatrix} \frac{-3}{5} \\ 1 \\ \frac{4}{5} \end{bmatrix}$ \\

Note that this vector is $\frac{4}{5}$ times the vector we found in (\ref{nullpractice:4}). \\ 

}

\sol{At this point, stress that the choice of free variable doesn't change the null space. Since subspaces are closed under scalar multiplication, the fact that this vector is a multiple of the previous shouldn't be a surprise to students. If it is, explain why this is the case.}

\ans{

Now, let's see what happens if we set $x=t$ instead. We can't use Equation (\ref{nullpractice:2}) yet, so let's try using Equation (\ref{nullpractice:1}). $t - y + 2z = 0$. Now what? How do we find the value for $y$ or $z$ in terms of $t$? ...We can't. So $x$ does not work...

}

\qitem{How can we know which variables can be used as free variables?}

\ans{Pick your free variables are by looking at columns with no pivots. Although, sometimes, other variables might work (like $y$ above), the variables with no pivots will always work!}

\qitem{
	Now consider another matrix, $\mathbf{C} = \begin{bmatrix} 1 & -2 & -6 & 12 \\ 2 & 4 & 12 & -17 \\ 1 & -4 & -12 & 22 \end{bmatrix}$
	Without doing any math, will this matrix have a trivial nullspace, i.e. consisting of only $\vec{0}$?
}

\ans{No! A 3x4 matrix can simply not have 4 pivots. So at least one of the variables will need to be free!}

\qitem{
	Consider another matrix, $\mathbf{D} = \begin{bmatrix} 1 & -2 & -6 & 12 \\ 0 & -2 & -6 & 10 \\ 0 & 0 & 0 & -1 \end{bmatrix}$. Find vector(s) that span the nullspace.
}

\ans{
	In terms of equations, let the variables for cols 1-4 be $a$ to $d$ respectively. Column 3 does not have a pivot. So $c$ is free. Let $c = t$. At this point, we should feel comfortable reading the matrix as its equations without explicitly writing the equations! \\

	Row 3 of the matrix says that $-d = 0$, or that $d = 0$. \\

	Row 2 says that $-2b -6c + 10d = 0 \implies -2b -6t = 0 \implies b=-3t$. \\

	Row 1 says that $a - 2b -6c + 12d = 0 \implies a + 6t -6t = 0 \implies a=0$. \\

	The vector that spans this nullspace is $\begin{bmatrix} 0 \\ -3 \\ 1 \\ 0\end{bmatrix}$

	}
\sol{
	Students could be confused about 'pivots'. Column 3 doesn't have a pivot because it has a 0 in the place of the 'diagonal' instead. Also, ask them which column doesn't have a pivot in this case: $\begin{bmatrix} 1 & -2 & -6 & 12 \\ 0 & -2 & -6 & 10 \\ 0 & 0 & -1 & 9 \end{bmatrix}$ (note that the last row is different.) Basically, column 4 doesn't a pivot, because that pivot would have been in the 4th row which doesn't exist. \textit{Make sure these concepts about pivots settle in}.

}
\qitem{Consider one final matrix, $\mathbf{E} = \begin{bmatrix} 1 & -2 & -6 & 12 \\ 0 & -2 & -6 & 10 \\ 0 & 0 & 0 & 0\end{bmatrix}$. What are the vector(s) that span this nullspace?}

\ans{
	Again, let the variables for the columns be $a$ to $d$ respectively. Columns 3 and 4 don't have pivots. So let's set both of them to be free! \\

	Let $c = t, d = s$. \\

	Row 2 says $-2b -6c + 10d = 0 \implies -2b = 6t - 10s \implies b = -3t + 5s$. \\

	Row 1 says $a -2b -6c + 10d = 0 \implies a = 0$. \\

	The general form of vectors in the nullspace is then $\begin{bmatrix} 0 \\ -3t + 5s \\ t \\ s \end{bmatrix}$. This needs to be rewritten by splitting the free variables $s\begin{bmatrix} 0 \\ 5 \\ 0 \\ 1 \end{bmatrix} + t\begin{bmatrix} 0 \\ -3 \\ 1 \\ 0\end{bmatrix}$. \\

	Finally we conclude that the vectors that span the nullspace are $\begin{bmatrix} 0 \\ 5 \\ 0 \\ 1 \end{bmatrix}$ and $\begin{bmatrix} 0 \\ -3 \\ 1 \\ 0 \end{bmatrix}$. \\

	Observation: notice that the number of free variables = number of columns without pivot = number of vectors required to span the nullspace = dimension of the nullspace!
}




\end{enumerate}



\end{qunlist}

\end{document}

